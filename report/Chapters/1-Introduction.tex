\chapter{Introduction}
\label{cp:introduction}


\section{Contexte et motivation}
\paragraph{Dans le domaine de l'ingénierie de contrôle, les systèmes de stabilisation sont essentiels pour maintenir l'équilibre de diverses structures et dispositifs mécaniques. Ces systèmes sont largement appliqués dans de nombreux domaines, notamment la robotique, l'aérospatiale, l'automobile et l'automatisation industrielle. La capacité à stabiliser un système de manière efficace peut considérablement améliorer ses performances et sa fiabilité.}
\paragraph{Ce projet se concentre sur la stabilisation d'une barre rotative à un degré de liberté, qui sert de modèle simplifié pour des problèmes de stabilisation plus complexes. Le système de la barre rotative, souvent appelé pendule inversé, est un problème classique en théorie de contrôle et fournit une plate-forme précieuse pour tester et développer des algorithmes de contrôle.}

\section{Objectifs}
\begin{enumerate}
	\item Concevoir la structure mécanique.
	\item Intégrer un \gls{imu} pour obtenir des données d'orientation en temps réel.
	\item Développer et mettre en œuvre un algorithme de contrôle PID pour traiter les données de l'\gls{imu} et contrôler les rotors.
\end{enumerate}

\newpage

\section{Portée}

\paragraph{La portée de ce projet comprend la conception et la mise en œuvre des composants matériels et logiciels nécessaires pour le système de stabilisation. Les éléments clés du projet sont :}

\begin{enumerate}
	\item \textbf{Etude théorique :} Cela implique la compréhension et l'analyse du système de la barre rotative, y compris les équations de mouvement, principes de contrôle et estimation de l'état.
	\item \textbf{Conception Mécanique :} Cela implique la conception et la construction de la barre rotative ainsi que le support fixe.
	\item \textbf{Conception Électronique :}  Cela couvre l'intégration de la carte d'aquisition de données et de controle y compris le câblage et la conception des circuits.
	\item \textbf{Conception Logiciel :} Cela comprend la programmation de l'algorithme de contrôle \acrshort{pid} et l'intégration de l'\gls{imu} pour obtenir des données d'orientation en temps réel.
\end{enumerate}
