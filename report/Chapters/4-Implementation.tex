\chapter{Development Logiciel}
\label{cp:implementation}

\section{Introduction}

\paragraph{Dans ce chapitre, nous allons présenter le développement logiciel du système de stabilisation. Nous allons commencer par présenter la configuration des différents composants électroniques, ensuite nous allons présenter l'implémentation de l'algorithme de contrôle PID.}

\paragraph{Le développement logiciel du système de stabilisation est basé sur l'utilisation de la bibliothèque Arduino pour la gestion des capteurs et des actionneurs. La bibliothèque Arduino fournit des fonctions prédéfinies qui facilitent la programmation des cartes Arduino.}

\paragraph{
	Pour faciliter la tache de développement, On definit des constantes pour tous les adresses registres de l'\gls{imu} utilises et les valeurs des plages de mesure. Par exemple pour les registres d'acceleration, de vitesse angulaire et de temperature.
}
\paragraph*{}
\begin{listing}[!htpb]
	\begin{minted}{cpp}
#define MPU6050_RA_ACCEL_XOUT_H     0x3B
#define MPU6050_RA_ACCEL_XOUT_L     0x3C
#define MPU6050_RA_ACCEL_YOUT_H     0x3D
#define MPU6050_RA_ACCEL_YOUT_L     0x3E
#define MPU6050_RA_ACCEL_ZOUT_H     0x3F
#define MPU6050_RA_ACCEL_ZOUT_L     0x40
#define MPU6050_RA_TEMP_OUT_H       0x41
#define MPU6050_RA_TEMP_OUT_L       0x42
#define MPU6050_RA_GYRO_XOUT_H      0x43
#define MPU6050_RA_GYRO_XOUT_L      0x44
#define MPU6050_RA_GYRO_YOUT_H      0x45
#define MPU6050_RA_GYRO_YOUT_L      0x46
#define MPU6050_RA_GYRO_ZOUT_H      0x47
#define MPU6050_RA_GYRO_ZOUT_L      0x48
	\end{minted}
	\caption{Registers Addresses}
	\label{listing:registers-addresses}
\end{listing}


\section{Composants Logiciels}

\paragraph{L'approche orientée objet est utilisée pour structurer le code en classes et en objets. Cela permet de regrouper les fonctions et les variables associées dans des classes, ce qui facilite la gestion et la maintenance du code.}

\paragraph{Le code est divisé en plusieurs fichiers pour faciliter la gestion et la maintenance du code. Les fichiers principaux sont :}

\subsection{MPUSensor Class}

\paragraph{La classe MPUSensor est utilisée pour gérer les données de l'\gls{imu}. Elle contient les fonctions pour initialiser l'\gls{imu}, lire les données d'accélération et de vitesse angulaire, filtrer et calculer l'orientation du système.}
\begin{listing}[!htpb]
	\inputminted{cpp}{Code/MPUSensor.h}
	\caption{Classe MPUSensor}
	\label{listing:mpu-sensor}
\end{listing}

\paragraph{Les fonctions elementaires sonts readBits et writeBits qui permet de lire et d'ecrire respectivement des bits dans un registre de l'\gls{imu}. Ils vont etre tres utiles. Ils sont implementées comme dessus.}
\begin{listing}[!htpb]
	\begin{minted}{cpp}
/**
 * @brief Read multiple bits from a register
 * 
 * @param reg Register address
 * @param data Pointer to the data buffer
 * @param length Number of bits to read
 */
void MPUSensor::readBits(uint8_t reg, uint8_t *data, uint8_t length) {
	Wire.beginTransmission(MPU_ADDR);
	Wire.write(reg);
	Wire.endTransmission(false);
	Wire.requestFrom(MPU_ADDR, length, true);
	for (uint8_t i = 0; i < length; i++) {
		data[i] = Wire.read();
	}
}
	\end{minted}
	\caption{Implementation de la fonction readBits}
	\label{listing:read-bits}
\end{listing}

\begin{listing}[!htpb]
	\begin{minted}{cpp}
/** 
 * @brief Write 8 bits to a register
 * 
 * @param reg Register address
 * @param data Data to write
 */
void MPUSensor::writeBits(uint8_t reg, uint8_t data) {
	Wire.beginTransmission(MPU_ADDR);
	Wire.write(reg);
	Wire.write(data);
	Wire.endTransmission(true);
}
	\end{minted}
	\caption{Implementation de la fonction writeBits}
	\label{listing:write-bits}
\end{listing}

\paragraph{Apres avoir implementer ces fonction on peut passer a:}
\paragraph*{}
\begin{listing}[!htpb]
	\begin{minted}{cpp}
void setFullScaleGyroRange(uint8_t range){
	writeBits(MPU6050_RA_GYRO_CONFIG, range);
}
	\end{minted}
	\caption{Implementation de la fonction setFullScaleGyroRange}
	\label{listing:set-full-scale-gyro-range}
\end{listing}

\paragraph*{De meme pour les fonctions setFullScaleAccelRange et setSleepMode.}

\subsubsection*{Calibration de l'IMU}

\paragraph{Pour calibrer l'\gls{imu}, on doit mesurer les valeurs de l'accélération et de la vitesse angulaire dans les trois axes X, Y et Z. Ensuite, on doit calculer les offsets pour chaque axe en utilisant les valeurs moyennes des mesures.}
\begin{listing}[!htpb]
	\begin{minted}{cpp}
/**
 * @brief Calibrate the MPU6050 with 500 samples
 */
void MPUSensor::calibrate(){
	int16_t raw_ax, raw_ay, raw_az, raw_gx, raw_gy, raw_gz;
	int32_t ax_sum = 0, ay_sum = 0, az_sum = 0, gx_sum = 0, gy_sum = 0, gz_sum = 0;
	for (int i = 0; i < 500; i++) {
		readAccelData(&raw_ax, &raw_ay, &raw_az);
		readGyroData(&raw_gx, &raw_gy, &raw_gz);
		ax_sum += raw_ax;
		ay_sum += raw_ay;
		az_sum += raw_az;
		gx_sum += raw_gx;
		gy_sum += raw_gy;
		gz_sum += raw_gz;
		delay(2);
	}
	ax_offset = ax_sum / 500;
	ay_offset = ay_sum / 500;
	az_offset = az_sum / 500;
	gx_offset = gx_sum / 500;
	gy_offset = gy_sum / 500;
	gz_offset = gz_sum / 500;
}
	\end{minted}
	\caption{Implementation de la fonction calibrate}
	\label{listing:mpu-calibrate}
\end{listing}

\newpage

\paragraph{Maintenant on definit la fonction setup qui va initialiser la configuration de l'\gls{imu} comme suit :}

\begin{itemize}
	\item \textbf{Plage de mesure de la vitesse angulaire :} $\pm 500$ °$/s$
	\item \textbf{Plage de mesure de l'accélération :} $\pm 4g$
	\item \textbf{Mode de fonctionnement :} Actif
\end{itemize}

\paragraph{Ensuite elle calibre l'\gls{imu}}

\begin{listing}[!htpb]
	\begin{minted}{cpp}
/**
 * @brief Setup the MPU6050 configuration
 */
void MPUSensor::setup() {
	setFullScaleGyroRange(MPU6050_GYRO_FS_500);
	setFullScaleAccelRange(MPU6050_ACCEL_FS_4);
	setSleepEnabled(false);
	calibrate();
}
	\end{minted}
	\caption{Implementation de la fonction setup}
	\label{listing:mpu-setup}
\end{listing}

\paragraph*{Pour la fonction reandAndUpdate :}

\begin{itemize}
	\item \textbf{Lire les données de l'\gls{imu} :} Accélération et vitesse angulaire
	\item \textbf{Soustrait bias des capteurs :} En soustrayant les offsets
	\item \textbf{Transforme d'unite :} De raw à g pour l'accélération et de raw à degrés par seconde pour la vitesse angulaire
	\item \textbf{Mettre à jour les valeurs :} Des variables de l'accélération et de la vitesse angulaire
	\item \textbf{Fusionner les données :} En utilisant le filtre de Kalman
\end{itemize}

\newpage

\begin{listing}[!htpb]
	\begin{minted}{cpp}
/**
 * @brief Read and update the orientation
 */
void MPUSensor::readAndUpdate() {
	int16_t raw_ax, raw_ay, raw_az, raw_gx, raw_gy, raw_gz;
	readAccelData(&raw_ax, &raw_ay, &raw_az);
	readGyroData(&raw_gx, &raw_gy, &raw_gz);
	raw_ax -= ax_offset;
	raw_ay -= ay_offset;
	raw_az -= az_offset;
	raw_gx -= gx_offset;
	raw_gy -= gy_offset;
	raw_gz -= gz_offset;
	
	ax = raw_ax * RAW_TO_G;
	ay = raw_ay * RAW_TO_G;
	az = raw_az * RAW_TO_G;

	gx = raw_gx * RAW_TO_DEG;
	gy = raw_gy * RAW_TO_DEG;
	gz = raw_gz * RAW_TO_DEG;

	float angle_acc = asin((float)raw_ax / 8200.0) * DEG_TO_RAD;
	float angle_gyro = angle + (float)raw_gy * RAW_TO_DEG * DT;

	
	angle = 0.9996 * angle_gyro + 0.0004 * angle_acc;
}
	\end{minted}
	\caption{Implementation de la fonction readAndUpdate}
	\label{listing:mpu-read-and-update}
\end{listing}

\newpage

\subsection{PID}

\paragraph{La classe PID est utilisée pour implémenter l'algorithme de contrôle PID. Elle contient les fonctions pour initialiser les paramètres du PID, calculer la commande de contrôle et ajuster les paramètres du PID.}

\begin{listing}[!htpb]
	\inputminted{cpp}{Code/PIDController.h}
	\caption{Classe PID}
	\label{listing:pid}
\end{listing}

\paragraph{La fonction update est utilisée pour calculer la commande de contrôle en utilisant l'algorithme de contrôle PID. Elle prend en entrée la consigne et la valeur actuelle, et retourne la commande de contrôle.}

\begin{listing}[!htpb]
	\begin{minted}{cpp}
/**
 * @brief Update the PID controller
 * 
 * @param setpoint Setpoint
 * @param input Input value
 * @return Command value
 */
 float update(float setpoint, float measuredValue, float deltaTime) {
        float error = setpoint - measuredValue;
        integral_ += error * deltaTime;
        float derivative = (error - previousError_) / deltaTime;
        float output = kp_ * error + ki_ * integral_ + kd_ * derivative;

        // Clamp output to min/max
        if (output > outputMax_) output = outputMax_;
        else if (output < outputMin_) output = outputMin_;

        previousError_ = error;
        return output;
    }
	\end{minted}
	\caption{Implementation de la fonction update}
	\label{listing:pid-update}
\end{listing}

\subsection{MotorsController}

\paragraph{La classe MotorsController est utilisée pour contrôler les moteurs. Elle contient les fonctions pour initialiser les moteurs, contrôler la vitesse des moteurs et arrêter les moteurs. elle utilise la classe PID pour commander le moteur.}
\paragraph*{}
\begin{listing}[!htpb]
	\inputminted{cpp}{Code/MotorsController.h}
	\caption{Classe MotorController}
	\label{listing:motor-controller}
\end{listing}

\paragraph{La fonction setup est utilisée pour initialiser les moteurs en utilisant la bibliothèque Servo.}

\paragraph{Lafonction update execute le calcul PID et met à jour la vitesse des moteurs.}

\begin{listing}[!htpb]
	\begin{minted}{cpp}
/**
 * @brief Update the motors speed
 */
void update(float setpoint, float measuredValue, float deltaTime) {
	float output = pid.update(setpoint, measuredValue, deltaTime);
	esc_1.writeMicroseconds(1500 + output);
	esc_2.writeMicroseconds(1500 - output);
}
	\end{minted}
	\caption{Implementation de la fonction update}
	\label{listing:motor-update}
\end{listing}

\subsection{Code Arduino}



\begin{listing}[!htpb]
	\begin{minted}{cpp}
	#include <Arduino.h>
	#include "stabilization.h"

	MPUSensor mpu;
	MotorsController motors;

	// The time variables to keep track of the time.
	double current_time = 0.0f, deltaTime = 0.004f;
	unsigned long start_micros, current_micros, deltat_micros;

	deltaTime_micros = deltat * MICRO_TO_SEC;

	void setup() {
		start_micros = micros();

		mpu.setup();
		motors.setup();
	}

	void loop() {
		current_micros = micros();
		current_time = current_micros / MICRO_TO_SEC; // Convert microseconds to seconds.

		mpu.readAndUpdate();
		motors.update(0, mpu.angle, deltaTime);

		// Wait for the next loop. 0.004s - time taken for the loop.
		while (micros() - current_micros < deltaTime_micros);
	}
	\end{minted}
	\caption{Code Arduino}
	\label{listing:arduino-code}
\end{listing}
