\chapter{Conception du Système}
\label{cp:systemdesign}

\section{Introduction}

\paragraph{Dans ce chapitre, nous allons présenter la conception du système. Nous allons commencer par présenter la conception mecanique du système, ensuite nous allons présenter la conception électronique du système.
}
\section{Conception Mécanique}

\subsection{Schema Simple}

\paragraph{
	Le système est principalement composé de deux parties : la partie mobile et la partie fixe. La partie mobile est composée de deux moteurs brushless qui sont montés sur une barre rotative. La partie fixe est composée d'une base qui supporte la barre rotative. La figure \ref{fig:system} montre le schéma simple du système.
}

\begin{figure}[!htpb]
	\centering
	\includegraphics[width=0.6\linewidth]{Figures/schema-simple.png}
	\caption{Schéma Simple du Système}
	\label{fig:system}
\end{figure}

\paragraph{Deux liaisons mécaniques seront utilisées. La première liaison est une liaison pivot qui permet à la barre rotative de tourner autour de l'axe vertical. La deuxième liaison est une liaison encastrement des moteurs à la barre rotative et du support à la base.}

\subsection{Modele 3D}

\paragraph{
	Pour assurer la fixation des moteurs à la barre rotative, on va percer les memes trous existants dans les moteurs dans les deux extrémités de la barre rotative. Ensuite, on va fixer les moteurs à la barre rotative en utilisant des vis.
}
\paragraph*{}
\begin{figure}[!htpb]
	\centering
	\includegraphics[width=0.6\linewidth]{Figures/motor-draw.jpg}
	\caption{Dessin technique du moteur}
	\label{fig:mobile}
\end{figure}

\paragraph{Pour assurer la fluidité de la rotation de la barre rotative, on va utiliser un roulement à billes pour la liaison pivot. Ce roulement à billes sera fixé à la base et à la barre rotative.}

\paragraph{La partie fixe est composée d'une base qui supporte la barre rotative et une table de bois.}
\paragraph*{}
\begin{figure}[!htpb]
	\centering
	\includegraphics[width=\linewidth]{Figures/3d-cad.png}
	\caption{Modele 3D du design mecanique}
	\label{fig:3d-model}
\end{figure}

\newpage
\paragraph{Après avoir couper et souder les différentes pièces metaliques pour former le modèle, on va obtenir le modèle mecanique comme montré ci-dessous.}
\paragraph*{}
\begin{figure}[!htpb]
	\centering
	\includegraphics[width=0.8\linewidth]{Figures/model-mec-1.jpeg}
	\caption{Modèle Mecanique 1}
	\label{fig:real-model}
\end{figure}

\begin{figure}[!htpb]
	\centering
	\includegraphics[width=0.8\linewidth]{Figures/model-mec-2.jpeg}
	\caption{Modèle Mecanique 2}
	\label{fig:real-model-2}
\end{figure}



\section{Conception Électronique}

\subsection{Carte d'aquisition et de controle}

\paragraph{
	On va utiliser une carte Arduino ATMega 2560 pour la gestion des capteurs et des actionneurs. Cette carte est équipée de plusieurs entrées/sorties numériques et analogiques qui nous permettent de connecter les différents capteurs et actionneurs.
}

\begin{figure}[!htpb]
    \centering
    \includegraphics[width=0.6\linewidth]{Figures/arduino.png}
    \caption{Arduino Mega 2560 Pins Layout}
    \label{fig:ArduinoMega2560}
\end{figure}

\paragraph{
	Cette carte peut communiquer en utilisant plusieurs protocoles de communication comme le protocole I2C ou SPI qui nous seront utile pour communiquer avec l'IMU. Aussi, cette carte contient des pins PWM qui nous permettent de contrôler les deux moteurs controlleurs.
}

\subsection{MPU 6050}

\paragraph{Le MPU 6050 est un capteur d'inertie qui combine un accéléromètre et un gyroscope. Il est utilisé pour mesurer l'accélération et la vitesse angulaire. Il communique avec la carte Arduino en utilisant le protocole I2C ou SPI.}

\begin{figure}[!htpb]
	\centering
	\includegraphics[width=0.7\linewidth]{Figures/MPU6050.png}
	\caption{MPU 6050}
	\label{fig:MPU6050}
\end{figure}

\subsubsection{Plages de mesures}

\paragraph{Le MPU 6050 peut mesurer l'accélération selon les trois axes X, Y et Z avec une plage de mesure de $\pm 2g$, $\pm 4g$, $\pm 8g$ et $\pm 16g$. Il peut aussi mesurer la vitesse angulaire selon les trois axes X, Y et Z avec une plage de mesure de $\pm 250$ °$/s$, $\pm 500$ °$/s$, $\pm 1000$ °$/s$ et $\pm 2000$ °$/s$.}

\subsubsection{Pins}

\paragraph{Le MPU 6050 contient 8 pins qui sont :}

\begin{itemize}
	\item VCC : Alimentation 3.3V ou 5V
	\item GND : Masse
	\item SDA : Données I2C
	\item SCL : Horloge I2C
	\item XDA : Données I2C
	\item XCL : Horloge I2C
	\item AD0 : Adresse I2C
	\item INT : Interruption
\end{itemize}

\paragraph{Pour utiliser le MPU on peut utiliser les pins SDA et SCL pour communiquer en utilisant le protocole I2C. Et on laisse AD0 flottant pour utiliser l'adresse x68 par défaut pour l'MPU 6050. Donc on va utiliser les pins SDA, SCL et VCC pour alimenter le MPU 6050 et GND.}

\begin{figure}[!htpb]
	\centering
	\includegraphics[width=0.7\linewidth]{Figures/mpu-pinning.png}
	\caption{MPU 6050 Pins}
	\label{fig:MPU6050Pins}
\end{figure}

\newpage

\subsection{Moteurs et leurs drivers}

\paragraph{On va utiliser deux moteurs brushless 1000KV pour la propulsion. Ces moteurs sont alimentés et commander par des ESC (Electronic Speed Controller) qui sont des drivers pour les moteurs brushless.}

\paragraph{Ces moteurs contients 3 fils (triphase) qui seront connectés aux ESC.}

\paragraph{Les \gls{esc} sont connectés à la carte Arduino en utilisant les pins PWM pour contrôler la vitesse des moteurs. et alimenter par une batterie LiPo 11.1V.}


\subsubsection{Pinning}

\begin{figure}[!htpb]
	\centering
	\includegraphics[width=0.5\linewidth]{Figures/esc-motor-pining.png}
	\caption{ESC Pinning}
	\label{fig:ESCPins}
\end{figure}

\subsection{Circuit Electrique}
\paragraph{Apres avoir vu chaque composant du système, nous allons maintenant voir comment les connecter ensemble. La figure \ref{fig:circuit-final} montre le cablage du système.}
\paragraph*{}
\begin{figure}[!htpb]
	\centering
	\includegraphics[width=0.9\linewidth]{Figures/final-circuit.png}
	\caption{Circuit Final du Système}
	\label{fig:circuit-final}
\end{figure}

\newpage
